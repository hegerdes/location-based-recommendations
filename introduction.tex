\documentclass[beamer]{beamer} % "beamer" zu "handout" ändern, wenn das Aufdecken ausgeschaltet werden soll

\usepackage[ngerman]{babel}
\usepackage[utf8]{inputenc}

\usepackage{bm}
\usepackage{tcolorbox}
\usepackage{setspace}

\usetheme[progressbar=frametitle]{metropolis}
\beamertemplatenavigationsymbolsempty
\setbeamertemplate{headline}{}
\setbeamertemplate{footline}{}

\newcommand{\userstory}[1]{
\begin{tcolorbox}[boxrule=0.2mm, top=0mm, left=0mm, bottom=1mm, title=User Story]
	\begin{singlespace}
		#1
	\end{singlespace}
\end{tcolorbox}
}


\title{Location Based Recommendations}
\author[H.~Gerdes, J.B.~Latzel, L.~Richardt]{Henrik Gerdes, Johannes B. Latzel, Leon Richardt}
\date[13.08.2018]{13. August 2018}

\begin{document}
	
	\begin{frame}
		\titlepage
	\end{frame}

	\begin{frame}{Definition}
		\begin{quote}
			Location-based recommendation is a recommender system that incorporates \textcolor{red}{location information}, such as that from a mobile device, into algorithms to attempt to provide more \textcolor{red}{relevant recommendations} to users. This could include recommendations for \textnormal{[\ldots]} events near the user's location.
		\end{quote}
		\hfill---~Wikipedia
		\vfill
		\bm{$\Rightarrow$} \textbf{Einbeziehung von Standortdaten in Empfehlungen}
	\end{frame}

	\begin{frame}{In diesem Projekt}
		Welche Daten könnte man für LBR nutzen?
		\begin{itemize}
			\item GPS-Position
			\item Event-Filter
			\item User-Teilnahme an ähnlichen Events in der Vergangenheit
			\item Popularität bestimmter Locations in der Umgebung (mithilfe historischer oder Echtzeit-Daten)
		\end{itemize}
	\end{frame}

	\begin{frame}{User Stories}
		\userstory{Ich möchte Zeit sparen, indem ich Veranstaltungen, die mich interessieren, schnell finden kann.}
	
		\userstory{Ich möchte festlegen können, für welche Art von Events ich benachrichtigt werde.}
	\end{frame}

	\begin{frame}{Projektziel}
		\textbf{Must-Haves:}
		\begin{itemize}
			\item Geordnete Liste und Hervorhebung passender Events (auch auf der Karte)
			\item Push-Nachrichten, wenn geeignete Events in der Nähe stattfinden
			\item Konfiguration von Präferenzen, Suchradius, Push-Nachrichten etc.
		\end{itemize}
		\uncover<2->{%
			\textbf{Nice to have:}
			\begin{itemize}
				\item content...
				\item more content...
			\end{itemize}
		}
	\end{frame}
	
\end{document}